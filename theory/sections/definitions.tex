\documentclass[../main.tex]{subfiles}

\begin{document}

\begin{definition}{Rule}{rule}
    A \emph{rule} is a non-increasing terminating sequence of non-negative integers where the first value is positive.
    The set of all rules is denoted by $\rules\subset\N_1\to\N_0$.
    A sequence $R$ is a rule iff
    \begin{align*}
        &&R[1]>&0\\
        &\forall k\in\N_1&R[k]\geq&R[k+1]\\
        &\exists n\in\N_1\forall k> n&R[k]=&0
    \end{align*}
\end{definition}

\begin{definition}{Rule length}{rule_length}
    The \emph{length} of a rule $R\in\rules$ is denoted $\rlength R$ and is defined as the greatest $n\in\N_1$ such that $R[n]>0$.
    In other words, it is the number of positive values in the rule.
\end{definition}

\begin{definition}{Rule base}{rule_base}
    The \emph{base} of a rule $R\in\rules$ is denoted $\base R\in\R$ and is defined as the sole positive root of the polynomial
    \begin{align*}
        -x^n+\sum_{k=1}^nR[k]x^{n-k}
    \end{align*}
    where $n=\rlength R$.
\end{definition}

\begin{theorem}{Rule base well-definedness}{rule_base_well_definedness}
    The base of a rule exists and is unique.
    More formally, for any rule $R\in\rules$ there exists exactly one positive real number $x\in\R$ such that
    \begin{align*}
        -x^n+\sum_{k=1}^nR[k]x^{n-k}=0
    \end{align*}
    where $n=\rlength R$.
\end{theorem}
\begin{proof}
    Missing
    \todo{proof missing}
\end{proof}

\begin{definition}{Tape}{tape}
    A \emph{tape} is a double sided terminating sequence of non-negative integers.
    The set of all tapes is denoted by $\tapes\subset\Z\to\N_0$.
\end{definition}

\begin{definition}{Tape length}{tape_length}
    The \emph{length} of a tape $T\in\tapes$ is denoted $\tlength T\in\N_0$ and is defined as the size of the smallest interval that contains all non-zero values of $T$.
    More formally, it is given by
    \begin{align*}
        \tlength T=\max\left\{k\in\Z\mid T[k]>0\right\}-\min\left\{k\in\Z\mid T[k]>0\right\} + 1
    \end{align*}
\end{definition}

\begin{definition}{Tape operations}{tape_operations}
    Addition and multiplication for tapes is defined so that for all $k\in\Z$
    \begin{align*}
        (T_1+T_2)[k]=&T_1[k]+T_2[k]\\
        (T_1 T_2)[k]=&\sum_{l=-\infty}^\infty T_1[l]T_2[k-l]
    \end{align*}
\end{definition}

\begin{definition}{Tape norm}{tape_norm}
    For a positive real number $p\in\R^+$ the \emph{$p$-norm} of a tape $T\in\tapes$ is denoted $\norm pT\in\R^+$ and is defined as
    \begin{align*}
        \norm pT=&\sum_{k=-\infty}^\infty T[k]p^k
    \end{align*}
\end{definition}

\begin{theorem}{Tape norm validity}{tape_norm_validity}
    For any positive real number $p\in\R^+$ the $p$-norm on $\tapes$ is finite for all $T\in\tapes$ and satisfies the norm axiums.   
\end{theorem}
\begin{proof}
    Trivial
\end{proof}

\begin{definition}{Tape value}{tape_value}
    For a rule $R\in\rules$ and a tape $T\in\tapes$ the \emph{$R$-value} of $T$, or the \emph{value of $T$ according to $R$} is denoted $\rvalue{R}{T}\in\R$ and is defined as
    \begin{align*}
        \rvalue{R}{T}=&\sum_{k=-\infty}^\infty T[k]\base{R}^k
    \end{align*}
\end{definition}

\begin{definition}{Rule-equality}{rule_equality}
    Two tapes $T_1,T_2\in\tapes$ are \emph{equal according to a rule} $R\in\rules$, denoted $\requal R{T_1}{T_2}$, iff $\rvalue{R}{T_1}=\rvalue{R}{T_2}$.
\end{definition}

\begin{definition}{Validity}{validity}
    A tape $T\in\tapes$ is \emph{valid} for a rule $R\in\rules$ iff all values of $T$ are less than the first value of $R$.
    More formally iff for all $k$
    \begin{align*}
        T[k]<R[1]
    \end{align*}
\end{definition}

\end{document}